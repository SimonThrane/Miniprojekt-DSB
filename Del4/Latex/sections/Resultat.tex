\chapter{Resultat og Diskussion}\label{ch:Resultat og Diskussion}

Høreapparatet blev kørt med et klip fra en nyhedsreporter i stormvejr. Formålet med dette var at prøve at se om, vi kunne sortere noget baggrundsstøj fra vinden fra signalet, så man bedre kunne høre, hvad nyhedsværten sagde. 
Der blev derfor arbejdet med forskellige vægtninger af de enkelte frekvensbånd for at finde ud af, hvilke frekvens-områder, der skulle henholdsvis dæmpes og forstærkes for, at man kunne høre nyhedsværten bedst muligt.\\
Der er i opgaven blevet valgt kun at lave 5 forskellige frekvensbånd, der er blevet vægtet. Disse frekvensbånd kunne udviddes eller justeres og vægtes forskelligt, så de kunne dække en persons personlige behov. Man kunne også have udviddet høreapparatet, så det indeholdt flere frekvensbånd, hvilket ville gøre, at man kunne justere de enkelte frekvenser mere præcis og dermed opnå en bedre tilpasning til personlige behov. Det er dog blevet valgt ikke at udvidde høreapparatet til mere end 5 frekvensbånd, da dette skal ses som en prototype og fremstilling af funktionalitet i et høreapparatet.\\
Som det også fremgår af koden for Høreapparatet er både det originale og manipulerede lydklip blevet afspillet for at finde ud af om filteringen virkede med hensyn til at dæmpe noget af baggrundsstøjen. Der var en lille forskel i niveauet på baggrundsstøjen mellem de to lydklip. Men  da en del af baggrundsstøjen lå i samme frekvensbånd som nyhedsværtens stemme var det næsten umuligt at sortere alt baggrundsstøjen fra.