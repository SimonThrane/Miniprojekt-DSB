\chapter{Resultat og Diskussion}\label{ch:Resultat og Diskussion}

Vores analyse af de forskellige signaler viste, hvilke frekvenser, der var dominerende i de enkelte signaler. Denne viden er vigtig at have, da den fortæller meget godt, hvad signalet indeholder. \\Man kan f.eks. ved brug af analysen af symfoni-signalet (fig.\ref{fig:Symfoni udglattet}) se, at det indeholder stor aktivitet i frekvensområdet 100Hz-1000Hz, hvilket stemmer meget godt overens med, hvad man kunne forvente af frekvenser i et sådan signal.

Man kan, hvis man sammenligner de to signaler: Bas (fig.\ref{fig:Bas udglattet}) og musikboxen (fig.\ref{fig:Musikbox udglattet}). Se, at signalet fra musikboxen indeholder flere høje frekvens end Bas-signalet, dette er helt forventelig. Man kan også se, at signalet fra vinglasset, der knipses på indeholder nogen meget præcise frekvenser modsat f.eks. signalet fra symfoni-orkesteret, der indeholder mange forskellige instrumenter med forskellige frekvenser.

Vores analysesystem har dermed givet os en bedre forståelse for, hvilke frekvenser, der er dominerende i de enkelte signaler. Dette kan f.eks. bruges til at bestemme om, der er en bolt i motoren, der er slidt og skal skiftes. Da man med tilstrækkelig stor erfaring inde for et enkelt område kan bruge de signaler til at opfange fejl og slitage.

I analysen er det valgt ikke at bruge vindues-funktionerne, der det ikke gav mening i forhold til de signaler der er blevet arbejdet med. De anvendte signalers frekvnesopløsning syntes at have været så høj, at der ikke sås nogen nævneværdig forbedring af lækageproblemer, eller støjproblemer, ved brug af  fx Hanning-vinduer. Dette er eksemplificeret ved figur: \ref{fig:Motor hanning} der iøvrigt i sit afsluttende drop til sidst ligner at dette signal tidligere har haft en lavere samplingsfrekvens.  
