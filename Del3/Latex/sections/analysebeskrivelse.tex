\chapter{Teori}\label{ch:Teroi}

I denne opgave er der brugt en række forskellige digitale filtre og redskaber til at implementere en Audio equalizer. De forskellige filtre og redskaber, der er arbejdet med i opgaven er:
\begin{itemize}
\item FIR-filtre
\item IIR-filtre
\item Hanning funktionen
\item Z-transformation
\end{itemize}

\subsection{Equalizer}
En Equalizer er en , den .\\
Equaliseren er blevet implementeret ved at lave 5 forskellige båndpas filtre, der opdeler Audio-signalet i 5 forskellige frekvens dele. Disse båndpas filtre er blevet designet ved brug af IIR- og FIR-filtre. 
Der er et blokdiagram for Equalizeren på figur \ref{fig:Aktivitetsdiagram for Equalizeren}.

\begin{figure}[H]
	\centering
	\includegraphics[width=150mm]{figures/Equalizer_flowchart.PNG}
	\caption{Aktivitetsdiagram for Audio Equalizeren}
	\label{fig:Aktivitetsdiagram for Equalizeren}
\end{figure}

Efter signalet er blevet opdelt i 5 forskellige frekvensbånd bliver hver enkelt af frekvens båndene vægtet på forskellige vis.

I det nedenstående afsnit følger en beskrivelse af de to forskellige filtre type. Man kan på nedenstående billede  se den grundlæggende forskel på de to filtre.




\subsection{Z-transformation}
Z-transformationen er en tidsdiskret variant af Laplace Transformation, der transformere et tidsdiskret signal til frekvensdomænet.

Det matematiske udtryk for Z-transformation ses i ligning \ref{eq:Z-transformation}.

\begin{equation}\label{eq:Z-transformation}
{H(z)} = \displaystyle\sum_{n=-\infty }^{\infty} {h(n)z^{-n}}
\end{equation}
 
\subsection{FIR-filtre}
FIR står for Finite Implulse Response, hvilket betyder, at der er et endeligt antal af impulssvar.
Dette kan man se, i ligning \eqref{eq:FIR}, der er diffrensligningen for et FIR filter.
\begin{equation}\label{eq:FIR}
{y(n)} = \displaystyle\sum_{k=0}^{M-1} {b_{k}*x(n-k)}
\end{equation}


\subsection{IIR-filtre}
IIR står for Infinite Impulse Response, hvilket betyder, at det har et uendeligt antal output, da filteret benytter sig af feedback fra tidligere output. Hvis man kigger på ligning \eqref{eq:IIR} kan man se den generelle differensligning for et IIR-filter.


\begin{equation}\label{eq:IIR}
{y(n)} = \displaystyle\sum_{k=0}^{M-1} {b_{k}*x(n-k)}+\displaystyle\sum_{l=1}^{N-1} {a_{l}*y(n-l)}
\end{equation}

Fordelen ved et IIR-filter i forhold til et FIR-filter, at de kan bruges til at forstærke et signal, de benytter sig også af færre udregninger og dermed mindre hukommelse. 
Ulemperne ved IIR-filtre er, at de i modsætning til FIR-filtre godt kan være ustabile. 
Dette kan ses af ligning \eqref{eq:IIR overforingsfunktion}, der viser overføringsfunktionen for et IIR-filter i Z-domænet. Ligningnen er fremkommet ved brug af Z-transformationen.

\begin{equation}\label{eq:IIR overforingsfunktion}
{H(z)} =\frac{Y(z)}{X(z)} =\frac{\displaystyle\sum_{k=0}^{N} {b(k)*z^{-k}}}{1-\displaystyle\sum_{k=1}^{M} {a(k)*z^{-k}}}
\end{equation}

Som fremgår af ligning \eqref{eq:IIR overforingsfunktion} indeholder overføringsfunktionen både poler og nulpunkter. Det gælder for IIR-filtre, at de er stabile, hvis alle deres poler har en magnitude, der er mindre end 1 (Polerne lægger i enhedscirklen).


Man kan se de anvendte funktioner og en skabellon til et program i det vedhæftede bilag i afsnit \ref{ch:Bilag}.



