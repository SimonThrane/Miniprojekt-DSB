\chapter{Audio Equalizer}\label{ch:Equalizer}
\section{Audio Equalizer}
Audio Equalizeren er som beskrevet i afsnit \ref{ch:Teori} bestående af 5 båndpas-filtre. Der opdeler input-signalet i 5 frekvensbånd, der bliver vægtet forskelligt før de bliver sat sammen til et output-signal.

Der er blevet lavet et Matlab program, hvor der først bliver foretaget en Diskret Fourrier Transformation på indgangssignalet for at finde ud af, hvilke frekvenser indgangsignalet indeholder.
Efterfølgende bliver inputsignalet kørt igennem den fremstillede Audio Equalizer. Efter det er blevet kørt igennem Equalizeren bliver output-signalet også Diskret Fourrier Transformeret, så der kan overskues, hvilke frekvenskarakteristika dette signal indeholder.
\section{Matlab kode for de forskellige funktioner}
FIR-Båndpass funktion:
\newline
\newline
FIR-weight funktion:
\newline
\newline
Equalizeren blev kørt med forskellige signaler, hvor der blev eksperimenteret med forskellige vægtning af de enkelte Båndpas-filtre samt frekvensbåndet de dækkede. Efter at havde eksperimenteret med de forskellige frekvensbånd blev det bestemt, at de enkelte frekvensbånd skulle dække følgende frekvensområder, der alle ligger i det hørbare frekvensområde:
\begin{itemize}
	\item Bånd 1 -
	\item Bånd 2 -
	\item Bånd 3 -
	\item Bånd 4 -
	\item Bånd 5 -
\end{itemize}

\section{Resultat}
I dette afsnit fremstilles resultaterne af at køre Audio Equalizeren på en række forskellige signaler.
