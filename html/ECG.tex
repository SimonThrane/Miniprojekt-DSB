
% This LaTeX was auto-generated from MATLAB code.
% To make changes, update the MATLAB code and republish this document.

\documentclass{article}
\usepackage{graphicx}
\usepackage{color}

\sloppy
\definecolor{lightgray}{gray}{0.5}
\setlength{\parindent}{0pt}

\begin{document}

    
    \begin{par}
$$e^{\pi i} + 1 = 0$$
\end{par} \vspace{1em}
\begin{verbatim}
clear; clc;

%Indl�s filen i variablen y ved brug af audioread:
load ECG

% ***** Roter matrixen 90 grader ********************************
B = rot90(ecg);

% ***** Udv�lg 1 signal *****************************************
x = B(1,:);

% ***** Antal samples og varighed i sekunder ********************
Fsample=500;
N = length(x);
Tlength = N/Fsample;

% ***** Beregn DFT p� signalet **********************************
X=fft(x,N);

% ***** Frekvensakse setup **************************************
delta_f = Fsample/N;
f_axis = [0:delta_f:Fsample-delta_f];

% ***** Definer hann funktionen (vinduet)***********************
w = hanning(N);

% ***** Gang hann funktionen p� funktionen y *******************
x_hann=x.*w';

W = fft(x_hann,N);

% ***** Udklat et signal ***************************************
[f_oct3, Xm] = oct_smooth(X, Fsample, 12, [1 100]);

ECG_oct12dB = 20*log10(abs((2/N)*Xm));

% ***** Plot ***************************************************
figure(1); clf
semilogx(f_axis(1:0.5*end),20*log10(abs((2/N)*X(1:0.5*(end)))))
xlabel('Frekvens i Herz')
ylabel('St�rrelse i dB ift. 1 Volt')
title('ECG DFT-signal')
grid on

figure(2); clf
semilogx(f_axis(1:0.5*end),20*log10(abs((2/N)*W(1:0.5*(end)))))
xlabel('Frekvens i Herz')
ylabel('St�rrelse i dB ift. 1 Volt')
title('ECG med Hanning vindue')
grid on

figure(3); clf
semilogx(f_oct3, ECG_oct12dB)
xlabel('Frekvens i Herz')
ylabel('St�rrelse i dB ift. 1 Volt')
title('Udglattet ECG signal')
grid on

figure(4); clf
plot(x)
xlabel('Samples')
ylabel('Amplitude (volt)')
title('Originalsignal ECG')
grid on
\end{verbatim}



\end{document}
    
